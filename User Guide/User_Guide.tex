\documentclass{article}

\usepackage[T1]{fontenc}
\usepackage[utf8]{inputenc}
\usepackage{lmodern}
\usepackage{listings}
\usepackage{fullpage}
	
\author{James Curran and Shahaan Hassan}
\title{ColdCoffeeScript User Guide}

\lstset{basicstyle=\ttfamily}
\begin{document}
\maketitle
ColdCoffeeScript, its interpreter and accompanying features, and this User Guide are in no way affiliated with or endorsed by popular University of Southampton lecturer Gennaro Parlato.  
%\tableofcontents % for a table of contents

\section{Introductory Haiku}

\begin{quote}
\textit{We have distilled the\\One and only Gennaro\\Into a language}
\end{quote}

\section{"Serious" Introduction}
ColdCoffeeScript is a strongly and statically typed imperative domain-specific language, created for the purpose of manipulating regular languages. The language prides orthogonality over syntactic sugar so the developers have DELIBERATELY AIMED for a minimal syntax. All functions and operators can be treated as pass-by-value. Variables are only changed when explicitly re-assigned by the programmer. Four data types are supported: integer, boolean, string and set (of strings). For details on supported operations, see the "Syntax" section.

\section{Features}
ColdCoffeeScript boasts many features, guaranteed to make programming simply delightful. These include:
\begin{itemize}
\item Informative error messages: Delicious Gennaro-themed dialogue, for lexing, input, type and runtime errors
\item Type checking: Rigorous type checker to whip those unruly variables into shape before they go into the program!
\item Programmer convenience: Single line comments - include supplementary literature in your program by surrounding part of a line with hashes, like so \#Shahaan remember to add more features under here\#
\end{itemize}
\section{Syntax}
\subsection{Ignored characters}
Any characters surrounded by \lstinline|#| symbols are treated as comments and ignored by the interpreter. Outside of these symbols, only whitespace and the newline character are ignored.
\subsection{Program format}
A ColdCoffeeScript program is a series of statements (see "Statements" below), separated by a semicolon. Line comments can appear on any line of the program.
\subsection{Types}
\begin{itemize}
\item ColdCoffeeScript integers (named \lstinline|int| in code) are specified by any sequence of the characters 0-9. Nota Bene: Integers are bounded by the OCaml integer standard - see caml.inria.fr for more information. We certainly did.
\item Booleans (\lstinline|bool| in code) are standard boolean types, denoted by \lstinline|true| and \lstinline|false|.
\item Strings (\lstinline|string| in code) are sequences of characters from the lowercase English alphabet. The exception is \lstinline|:|, which represents an empty string. Strings are encased in double quotes in the program, but not in program input or output.
\item Sets (\lstinline|set| in code) are ordered collections of strings. A set cannot contain two identical strings, and the set is reordered automatically if necessary. Strings in sets follow alphabetical order, and if : is in the set it will be at the start.
\end{itemize}
\subsection{Arithmetic Operators}
All arithmetic operators are infix and only involve integers. The standard \lstinline|+|, \lstinline|-|, \lstinline|*| and \lstinline|/| are used, as well as \lstinline|<|, \lstinline|==| and \lstinline|>| for comparison. If two integers do not divide evenly, an error will occur.\\
\subsection{Boolean Operators}
All boolean operators are infix. The operators are the standard \lstinline|not|, \lstinline|and| and \lstinline|or|, with \lstinline|==| to check for equality.
\subsection{String Operators}
All string operators are infix. \lstinline|+| is used to concatenate strings. \lstinline|==| is used to check for equality.
\subsection{Set Operators}
The set operators are infix. ColdCoffeeScript provides \lstinline|union|, \lstinline|intersect| and \lstinline|concat|. Also, \lstinline|<set A> difference <set B>| is used to remove all values in set B from set A.
\subsection{Other Operators}
\lstinline|<string A> memberOf <set B>| is a boolean operator that checks if A is a member of B.\\
\lstinline|display <set A> <integer B>;| will output the first B elements of A.
\subsection{Variables}
A variable is created by stating its type, then name, followed by an equals sign, followed by the desired value. This is a statement so a semicolon is also needed.
\begin{itemize}
\item \lstinline|int MYNUMBER = 41;|
\item \lstinline|bool MYBOOL = true;|
\item \lstinline|string MYSTRING = "yawn";|
\item \lstinline|set MYSET = {"are", "you", "bored", "yet"};|
\end{itemize}
The variable's value is reassigned with the name, followed by an equals sign, followed by the new value, followed by a semicolon: \\
\lstinline|MYSET = {"god", "yes","someone","help","me","please"};| \\
It is of course also possible to (re-)assign the result of some expression to a variable, provided the result is of the type the variable was declared with: \\
\lstinline|MYSET = {"this", "joke", "has", "run"} union {"its", "co"+"urse"};| \\
Nota Bene: variable names can only consist of capital letters. This was DONE BY CHOICE to emphasise the importance of good variable naming.
\subsection{Using Input}
Input follows a strict structure where each line must be a set of strings, except for the last line which must be a number. Whitespace is ignored. For example:
\begin{lstlisting}
{a, b, c,       d}
{:}
{words}
4
\end{lstlisting}
Input is then accessed through pre-assigned variables at the start of the program. \lstinline|L1| is the language (set of strings) on line 1 of input, \lstinline|L2| on line 2 and so on. \lstinline|K| is the number at the end. These variables can be reassigned in code, but their original values cannot be retrieved unless they were saved elsewhere.
\subsection{Conditionals and Iteration}
This is a very brief overview of the if(-else) and while constructs in ColdCoffeeScript. See "Programming with ColdCoffeeScript" for more examples.
\begin{lstlisting}
if <bool> do
    <body>
end;

if <bool> do
    <body1>
else
    <body2>
end;

while <bool> do
    <body>
end;
\end{lstlisting}
\lstinline|<bool>| is an expression that evaluates to a boolean. \lstinline|<body>|, \lstinline|<body1>| and \lstinline|<body2>| are series of statements, just like the overall program. For an example of conditionals, refer to the FizzBuzz program in the Appendix. For an example of iteration, refer to the Kleene Star program.\\
Indentation is not necessary; ColdCoffeeScript code can take any shape desired by the programmer. To encourage this, code which uses whitespace to be shaped like a steaming cup of Italian coffee is permitted by the interpreter to run 50\% faster.\footnote{Completely untrue}
\section{Appendix: Example Programs}
\subsection{Hello World}
This is the anatomy of a Hello World Program in ColdCoffeeScript.
\begin{lstlisting}
display {"hello","world"} 2;
\end{lstlisting}
On line 1, \lstinline|display| is the function used to output a set to standard output. The output can probably be redirected as well.

On line 1, \lstinline|{"hello","world"}| is an example of what is known in ColdCoffeeScript as a set literal. The parentheses show the start and end of the set, and the set contains the two strings "hello" and "world", separated by a comma.

On line 1, \lstinline|2| is the number of elements of the set that will be printed. This is useful for very large sets where you don't care about all of the output.

The semicolon ends the display statement. If you do not end each statement with a semicolon, your statement will be just a concept.

The output of the program is as follows:
\begin{lstlisting}
{ciao,mondo}
\end{lstlisting}
This is because the language automatically translates English strings to Italian.
\newpage
\subsection{Not Really}
The actual output is:
\begin{lstlisting}
{hello,world}
\end{lstlisting}
\subsection{FizzBuzz}
The best FizzBuzz algorithms in other languages have \(\Omega (n)\) complexity; however, thanks to the properties of languages, the complexity in ColdCoffeeScript can be greatly reduced. Firstly, numbers are not permitted in languages so there is no need to write code to add integers to languages. Secondly, multiple instances of "Fizz", "Buzz" or "FizzBuzz" will be reduced to a single instance of each in the final set. We now therefore present an \(O(1)\) ColdCoffeeScript FizzBuzz program, which will produce the same output as any other FizzBuzz program in ColdCoffeeScript:
\begin{lstlisting}
if K > 14 do
    display {"buzz", "fizz", "fizzbuzz"} 3;
else if K > 4 do
    display {"buzz", "fizz"} 2;
else if K > 2 do
    display {"buzz"} 1;
else
    display {} 1;
end;
end;
end;
\end{lstlisting}
\subsection{Kleene Star}
There is no built-in function to generate the asterate of a language, but code can be made with relative ease.
\begin{lstlisting}
#Generate first K elements of "a"*#
int I = 0;
set STAR = {":"};
while I < K do
   STAR = ({"a"} concat STAR) union {":"};
   I = I + 1;
end;
display STAR K;       
\end{lstlisting}
\end{document}

